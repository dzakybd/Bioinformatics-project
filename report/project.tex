\documentclass[a4paper,oneside]{article}
\usepackage{indentfirst}
\usepackage{graphicx}
\usepackage{newtxtext,newtxmath}
\usepackage[top=0.7in, bottom=0.7in, left = 0.7in, right = 0.7in]{geometry}
\usepackage{enumitem}
\usepackage{caption}
\setlist{nolistsep}
\setlength{\belowcaptionskip}{-10pt}
\captionsetup[figure]{labelformat=empty}
\begin{document}
\title{\vspace{-0.7in}Term Project Proposal\\
Cancer Prediction using Single Nucleotide Polymorphism Dataset}
\author{Tanjung Dion (201899213)\\Fawwaz Dzaky Zakiyal (201899213)\\}
\date{Bioinformatics (Fall 2018)}
\maketitle
 
\section{Background}
The RFID based indoor positioning usually implemented in indoor object tracking, flight baggage handling, etc. The process start with a RFID reader detects a RFID tag when the object with the tag enters the reader’s detection range. But, often the recorded data inherent uncertainty, including noise/cross readings (it must be detected by a reader, but it detected by multiple readers) and incompleteness/missing readings (it must be detected by a reader, but it did not detected). Thus, the reading results are considered unclean and we need to cleansing this indoor RFID tracking data by reducing the noise, and recovering the incompleteness.

\section{Problem Statement}
Modeling of indoor RFID trajectory data with uncertainties using IR-MHMM;

\section{Problem Scope}
It compare three Learned models.

\section{Related Works}
It compare three Learned models.

\section{Methodology}
It compare three Learned models.

\begin{thebibliography}{1}
\bibitem{} A. I. Baba, H. Lu, T. B. Pedersen, and X. Xie. {\em Handling false
negatives in indoor RFID data.} In MDM, pages 117–126, 2014.
\bibitem{} B. Fazzinga, S. Flesca, F. Furfaro, and F. Parisi. {\em Cleaning trajectory data of RFID-monitored objects through conditioning under integrity constraints.} In EDBT, pages
379–390, 2014.
\end{thebibliography}
\end{document}